\documentclass[a4paper, 10pt]{article}

\usepackage[margin=1in]{geometry}
\usepackage{amsfonts, amsmath, amssymb, amsthm}
\usepackage[utf8]{inputenc}
\usepackage[english, main=ukrainian]{babel}
\usepackage{pgfplots}
\usepackage{bm}
\usepackage{physics}
\usepackage[unicode]{hyperref}
\usepackage{tikz-cd}
\usepackage{enumitem}
\usepackage{graphicx}
\usepackage{pgfplots}
\usepackage{pdfpages}
\usepackage{caption}
\usepackage{float}

\usepgfplotslibrary{fillbetween}

\usetikzlibrary{spy}
\usetikzlibrary{fit,matrix}
\usetikzlibrary{babel}

\def\rightproof{$\boxed{\Rightarrow}$ }

\def\leftproof{$\boxed{\Leftarrow}$ }

\newtheoremstyle{theoremdd}
  {\topsep}
  {\topsep}
  {\normalfont}
  {0pt}
  {\bfseries}
  {}
  { }
  {\thmname{#1}\thmnumber{ #2}\textnormal{\thmnote{ \textbf{#3}\\}}}

\theoremstyle{theoremdd}
\newtheorem{theorem}{Theorem}[subsection]
\newtheorem{definition}[theorem]{Definition}
\newtheorem{example}[theorem]{Example}
\newtheorem{proposition}[theorem]{Proposition}
\newtheorem{remark}[theorem]{Remark}
\newtheorem{lemma}[theorem]{Lemma}
\newtheorem{corollary}[theorem]{Corollary}

\newcommand\thref[1]{\textbf{Th.~\ref{#1}}}
\newcommand\defref[1]{\textbf{Def.~\ref{#1}}}
\newcommand\exref[1]{\textbf{Ex.~\ref{#1}}}
\newcommand\prpref[1]{\textbf{Prp.~\ref{#1}}}
\newcommand\rmref[1]{\textbf{Rm.~\ref{#1}}}
\newcommand\lmref[1]{\textbf{Lm.~\ref{#1}}}
\newcommand\crlref[1]{\textbf{Crl.~\ref{#1}}}

\renewcommand{\qedsymbol}{$\blacksquare$}


\makeatletter
\renewenvironment{proof}[1][Proof.\\]{\par
\pushQED{\hfill \qed}%
\normalfont \topsep6\p@\@plus6\p@\relax
\trivlist
\item\relax
{\bfseries
#1\@addpunct{.}}\hspace\labelsep\ignorespaces
}{%
\popQED\endtrivlist\@endpefalse
}
\makeatother

\DeclareMathOperator{\Ob}{Ob}
\DeclareMathOperator{\Hom}{Hom}
\DeclareMathOperator{\Set}{Set}

\title{Теорія категорії \\ І курс магістратура, 2 семестр}
    	
\begin{document}
\maketitle
\newpage
%\tableofcontents
%\newpage
%Section 1
%\section{Категорії}
\subsection{Основні означення}
\begin{definition}
\textbf{Категорія} $C$ складається з наступних компонент:
\begin{itemize}[nosep, wide=0pt, label={--}]
\item із набору \textbf{об'єктів}; об'єкти позначають за $X,Y,Z,\dots$, а набір позначають за $\Ob(C)$;
\item із набору \textbf{морфізмів}; морфізми позначають за $f,g,h,\dots$, а набір позначають за  $\Hom(C)$;
\item кожний морфізм має \textbf{область визначення} та \textbf{область значень}; позначається зазвичай як $f \colon X \to Y$, де об'єкт $X$ -- область визначення, об'єкт $Y$ -- область значень;
\item кожний об'єкт $X$ має \textbf{тотожний морфізм} $1_X \colon X \to X$;
\item для кожних морфізмів $f \colon X \to Y,\ g \colon Y \to Z$ існуватиме \textbf{композиція морфізмів} $g \circ f \colon X \to Z$.
\end{itemize}
При цьому всьому зобов'язані виконуватися такі аксіоми:
\begin{center}
\begin{enumerate}[nosep,wide=0pt,label={\arabic*)}]
\item для всіх морфізмів $f \colon X \to Y$ виконано $1_Y \circ f = f \circ 1_X = f$;
\item для кожних трьох морфізмів $f \colon W \to X, g \colon X \to Y, h \colon Y \to Z$ виконується асоціативність композиції, тобто $f \circ (g \circ h) = (f \circ g) \circ h$.
\end{enumerate}
\end{center}
\end{definition}

\begin{remark}
Морфізми ще часто називають \textbf{стрілочками}.
\end{remark}

\begin{example}
Розглянемо $\Set$ -- це буде категорія, яка складається з наступного:
\begin{itemize}[nosep, wide=0pt, label={--}]
\item $\Ob(\Set)$ -- набір всіх множин;
\item $\Hom(\Set)$ -- набір всіх відображень;
\item тотожне відображення $1_X \colon X \to X$ задається як $x \mapsto x$;
\item композиція між $f \colon X \to Y$ та $g \colon Y \to Z$ задається $g \circ f$ таким чином: $x \mapsto f(x) \mapsto g(f(x))$.
\end{itemize}
Ясно, що всі ці дві аксіоми виконані.
\bigskip \\
Важливо, що $\Ob(\Set)$ -- це саме \underline{набір} всіх множин, а не множина всіх множин. Тому що парадокс Рассела стверджує, що не існує множини, елементи яких будуть множинами.\\
До речі, $\Set(X,Y)$ -- набір всіх відображень $f \colon X \to Y$ -- буде, насправді, \underline{множиною}. Відображення між двома множинами -- це просто підмножина декартового добутку $X \times Y$. Коли ми беремо дві довільні множини $X,Y$, то звідси $X \times Y$ теж буде множиною.
\end{example}

\begin{example}
Розглянемо стисло ще приклади категорій:
\begin{enumerate}[nosep,wide=0pt,label={\arabic*)}]
\item $\text{Grp}$ -- об'єктами будуть групи; стрілками будуть гомоморфізми груп;
\item $\text{Ring}$ -- об'єктами будуть кільця; стрілками будуть гомоморфізмами кілець;
\item $\text{Top}$ -- об'єктами будуть топологічні простори; стрілочками будуть неперервні відображення;
\item $\text{Man}$ -- об'єктами будуть гладкі многовиди; стрілочками будуть гладкі відображення.
\end{enumerate}
\end{example}

\begin{example}
Розглянемо моноїд $M$. Ми можемо утворити категорію $\mathcal{M}$, яка містить єдиний об'єкт -- це моноїд.
\end{example}

\begin{example}
Розглянемо так званий посет $(P,\prec)$ (partially ordered set). Скажемо, що $\Ob(P) = P$ та $P(i,j)$ -- це будуть тільки ті стрілки, для яких $i \prec j$. Композиція тут існує, оскільки $\prec$ є транзитивним відношенням. Також існує тотожне відображення, оскільки $\prec$ є рефлексивним відношенням.\\
\textit{Навіть не обов'язково тут вимагати, щоб для $(P,\prec)$ відношення $\prec$ було антисиметричним.}
\end{example}

\begin{definition}
Задано $C$ -- категорія.\\
Стрілочка $f \colon X \to Y$ називається \textbf{ізоморфізмом}, якщо існує стрілка $g \colon Y \to X$, для якої
\begin{align*}
f \circ g = 1_Y \qquad g \circ f = 1_X
\end{align*}
У свою чергу об'єкти $X,Y$ даної категорії називаються \textbf{ізоморфними}.\\
Позначення: $X \cong Y$.
\end{definition}

\begin{definition}
\textbf{Ендоморфізмом} назвемо стрілочку $f \colon X \to X$. Тобто це стрілка між двома однаковими об'єктами.\\
\textbf{Автоморфізмом} назвемо ізоморфім $f$, який є ендоморфізмом.
\end{definition}

\begin{definition}
Категорія $C$ називається \textbf{дискретною}, якщо
\begin{align*}
C(A,B) = \begin{cases} \emptyset, & A \neq B \\ \{1_A\}, & A = B \end{cases}
\end{align*}
Тобто існують лише стрілки $A \to A$, і тільки тотожні.
\end{definition}

\begin{definition}
Категорія $D$ називається \textbf{підкатегорією} $C$, якщо
\begin{align*}
\text{набір об'єктів $D$ міститься в наборі об'єктів $C$} \\
\text{набір стрілок $A \to B$ в $D$ міститься в наборі стрілок $A \to B$ в $C$ для довільних об'єктів $A,B$ із $D$} \\
\text{композиція двох морфізмів в $D$ задається так само, як і в $C$}
\end{align*}
\end{definition}

\begin{definition}
Підкатегорія $D$ категорії $C$ називається \textbf{повною}, якщо
\begin{align*}
\text{набір стрілок $A,B$ в $D$ збігається з набором стрілок $A,B$ в $C$, для довільних об'єктів $A,B$ із $D$}
\end{align*}
\end{definition}

\subsection{Узагальнення ін'єкції та сюр'єкції}
\subsubsection{Монік}
\begin{definition}
Задано $C$ -- категорія.\\
Морфізм $\alpha \colon x \to y$ називається \textbf{моніком}, якщо
\begin{align*}
\alpha \beta_1 = \alpha \beta_2 \implies \beta_1 = \beta_2
\end{align*}
Тобто морфізм -- монік, якщо можна завжди скоротити зліва.
\begin{figure}[H]
\centering
\begin{tikzcd}
z \arrow[r, shift left, "\beta_2"] \arrow[r, shift right, "\beta_1" {yshift=-10pt}] & x \arrow{r}{\alpha} & y
\end{tikzcd}
\end{figure}
\end{definition}

\end{document}