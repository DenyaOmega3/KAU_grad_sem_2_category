\documentclass[a4paper, 10pt]{article}

\usepackage[margin=1in]{geometry}
\usepackage{amsfonts, amsmath, amssymb, amsthm}
\usepackage[utf8]{inputenc}
\usepackage[english, main=ukrainian]{babel}
\usepackage{pgfplots}
\usepackage{bm}
\usepackage{physics}
\usepackage[unicode]{hyperref}
\usepackage{tikz-cd}
\usepackage{enumitem}
\usepackage{graphicx}
\usepackage{pgfplots}
\usepackage{pdfpages}
\usepackage{caption}
\usepackage{float}

\usepgfplotslibrary{fillbetween}

\usetikzlibrary{spy}
\usetikzlibrary{fit,matrix}
\usetikzlibrary{babel}

\def\rightproof{$\boxed{\Rightarrow}$ }

\def\leftproof{$\boxed{\Leftarrow}$ }

\newtheoremstyle{theoremdd}
  {\topsep}
  {\topsep}
  {\normalfont}
  {0pt}
  {\bfseries}
  {}
  { }
  {\thmname{#1}\thmnumber{ #2}\textnormal{\thmnote{ \textbf{#3}\\}}}

\theoremstyle{theoremdd}
\newtheorem{theorem}{Theorem}[subsection]
\newtheorem{definition}[theorem]{Definition}
\newtheorem{example}[theorem]{Example}
\newtheorem{proposition}[theorem]{Proposition}
\newtheorem{remark}[theorem]{Remark}
\newtheorem{lemma}[theorem]{Lemma}
\newtheorem{corollary}[theorem]{Corollary}

\newcommand\thref[1]{\textbf{Th.~\ref{#1}}}
\newcommand\defref[1]{\textbf{Def.~\ref{#1}}}
\newcommand\exref[1]{\textbf{Ex.~\ref{#1}}}
\newcommand\prpref[1]{\textbf{Prp.~\ref{#1}}}
\newcommand\rmref[1]{\textbf{Rm.~\ref{#1}}}
\newcommand\lmref[1]{\textbf{Lm.~\ref{#1}}}
\newcommand\crlref[1]{\textbf{Crl.~\ref{#1}}}

\renewcommand{\qedsymbol}{$\blacksquare$}


\makeatletter
\renewenvironment{proof}[1][Proof.\\]{\par
\pushQED{\hfill \qed}%
\normalfont \topsep6\p@\@plus6\p@\relax
\trivlist
\item\relax
{\bfseries
#1\@addpunct{.}}\hspace\labelsep\ignorespaces
}{%
\popQED\endtrivlist\@endpefalse
}
\makeatother

\DeclareMathOperator{\Ob}{Ob}
\DeclareMathOperator{\Hom}{Hom}

\DeclareMathOperator{\Set}{\textbf{Set}}
\DeclareMathOperator{\FinSet}{\textbf{FinSet}}
\DeclareMathOperator{\Grp}{\textbf{Grp}}
\DeclareMathOperator{\Ab}{\textbf{Ab}}
\DeclareMathOperator{\Ring}{\textbf{Ring}}
\DeclareMathOperator{\Rng}{\textbf{Rng}}
\DeclareMathOperator{\Top}{\textbf{Top}}
\DeclareMathOperator{\Man}{\textbf{Man}}
\DeclareMathOperator{\Mod}{\textbf{Mod}}
\DeclareMathOperator{\Met}{\textbf{Met}}
\DeclareMathOperator{\Mani}{\textbf{Man}}

\title{Теорія категорії \\ І курс магістратура, 2 семестр}
    	
\begin{document}
\maketitle
\newpage
%\tableofcontents
%\newpage
%Section 1
%\section{Категорії}
\subsection{Основні означення}
\begin{definition}
\textbf{Категорія} $C$ складається з наступних компонент:
\begin{itemize}[nosep, wide=0pt, label={--}]
\item із набору \textbf{об'єктів}; об'єкти позначають за $x,y,z,\dots$, а набір позначають за $\Ob(C)$;
\item із набору \textbf{морфізмів із $x$ в $y$} $C(x,y)$ для всіх $x,y \in C$; морфізми позначають за $\alpha,\beta,\gamma,\dots$. Позначення $\alpha \colon x \to y$ або $x \overset{\alpha}{\to} y$ означають $\alpha$ -- морфізм із $x$ в $y$; ми називаємо $x$ \textbf{джерелом} та $y$ \textbf{ціллю};
\item кожний об'єкт $x$ має \textbf{тотожний морфізм} $1_x \colon x \to x$;
\item для кожних морфізмів $\alpha \colon x \to y,\ \beta \colon y \to z$ існуватиме \textbf{композиція морфізмів} $\beta \alpha \colon x \to z$.
\end{itemize}
При цьому всьому зобов'язані виконуватися такі аксіоми:
\begin{center}
\begin{enumerate}[nosep,wide=0pt,label={\arabic*)}]
\item для всіх морфізмів $\alpha \colon x \to y$ виконано $1_y \circ \alpha = \alpha \circ 1_x = \alpha$;
\item для кожних трьох морфізмів $\alpha \colon w \to x, \beta \colon x \to y, \gamma \colon y \to z$ виконується асоціативність композиції, тобто $\alpha (\beta \gamma) = (\alpha \beta) \gamma$.
\end{enumerate}
\end{center}
\end{definition}

\begin{remark}
Морфізми ще часто називають \textbf{стрілочками}.
\end{remark}

\begin{remark}
Морфізм $1_x$ для кожного об'єкта $x$ -- єдиний.
\end{remark}

\begin{example}
Розглянемо $\Set$ -- це буде категорія, яка складається з наступного:
\begin{itemize}[nosep, wide=0pt, label={--}]
\item $\Ob(\Set)$ -- набір всіх множин;
\item $\Hom(\Set)$ -- набір всіх функцій;
\item тотожне відображення $1_X \colon X \to X$ задається як $x \mapsto x$;
\item композиція між $f \colon X \to Y$ та $g \colon Y \to Z$ задається $g \circ f$ таким чином: $x \mapsto f(x) \mapsto g(f(x))$.
\end{itemize}
Ясно, що всі ці дві аксіоми виконані.
\bigskip \\
Важливо, що $\Ob(\Set)$ -- це саме \underline{набір} всіх множин, а не множина всіх множин. Тому що парадокс Рассела стверджує, що не існує множини, елементи яких будуть множинами.\\
До речі, $\Set(X,Y)$ -- набір всіх відображень $f \colon X \to Y$ -- буде, насправді, \underline{множиною}. Відображення між двома множинами -- це просто підмножина декартового добутку $X \times Y$. Коли ми беремо дві довільні множини $X,Y$, то звідси $X \times Y$ теж буде множиною.
\end{example}

\begin{example}
Розглянемо стисло ще приклади категорій:
\begin{figure}[H]
\centering
\begin{tabular}{c|c|c}
Категорія & Об'єкти & Морфізми \\
$\Grp$ & групи & гомоморфізми груп \\
$\Ab$ & абелеві групи & гомоморфізми груп \\
$\Rng$ & кільця & гомоморфізми кілець \\
$\Ring$ & кільця з одиницею & гомоморфізм кілець, що зберігають одиницю \\
${}_{R}\Mod$ & $R$-модуль & $R$-лінійне відображення \\
$\Top$ & топологічні простори & неперервній відображення \\
$\Met$ & метричні простори & неперервні відображення \\
$\Mani$ & гладкі многовиди & гладкі відображення
\end{tabular}
\end{figure}
\end{example}

\begin{example}
Можна представити категорію за допомогою графів. Категорія \textbf{0} буде взагалі порожньо виглядати. Категоріїя \textbf{1}, категорія \textbf{2}, категорія \textbf{3} виглядають таким чином:
\begin{figure}[H]
\centering
\begin{tikzcd}
\text{\textbullet} \ar[looseness=4, out=210, in=120]
\end{tikzcd}
\qquad
\begin{tikzcd}
\text{\textbullet} \arrow{r}{} \ar[looseness=4, out=210, in=120] & \text{\textbullet} \ar[looseness=4, out=60, in=-30]
\end{tikzcd}
\qquad
\begin{tikzcd}
\text{\textbullet} \arrow{rd}{} \arrow{rr}{} \ar[looseness=4, out=210, in=120] && \text{\textbullet} \ar[looseness=4, out=60, in=-30] \\
& \text{\textbullet} \ar[looseness=4, out=-45, in=225] \arrow{ru}{}
\end{tikzcd}
\end{figure}
\noindent
Так само є категорії \textbf{4},\textbf{5}, \dots
\end{example}

\begin{example}
Розглянемо моноїд $M$. Ми можемо утворити категорію $\mathcal{M}$, яка містить єдиний об'єкт -- це моноїд.
\end{example}

\begin{example}
Розглянемо так званий посет $(P,\prec)$ (partially ordered set). Скажемо, що $\Ob(P) = P$ та $P(i,j)$ -- це будуть тільки ті стрілки, для яких $i \prec j$. Композиція тут існує, оскільки $\prec$ є транзитивним відношенням. Також існує тотожне відображення, оскільки $\prec$ є рефлексивним відношенням.\\
\textit{Навіть не обов'язково тут вимагати, щоб для $(P,\prec)$ відношення $\prec$ було антисиметричним.}
\end{example}

\begin{definition}
Категорія $C$ називається \textbf{дискретною}, якщо
\begin{align*}
C(x,y) = \begin{cases} \emptyset, & x \neq y \\ \{1_x\}, & x = y \end{cases}
\end{align*}
Тобто існують лише стрілки $x \to x$, і тільки тотожні.
\end{definition}

\begin{definition}
Категорія $D$ називається \textbf{підкатегорією} $C$, якщо
\begin{align*}
\text{набір об'єктів $D$ міститься в наборі об'єктів $C$} \\
\text{набір стрілок $x \to y$ в $D$ міститься в наборі стрілок $x \to y$ в $C$ для довільних об'єктів $x,y$ із $D$} \\
\text{композиція двох морфізмів в $D$ задається так само, як і в $C$}
\end{align*}
\end{definition}

\begin{definition}
Підкатегорія $D$ категорії $C$ називається \textbf{повною}, якщо
\begin{align*}
\text{набір стрілок $x,y$ в $D$ збігається з набором стрілок $x,y$ в $C$, для довільних об'єктів $x,y$ із $D$}
\end{align*}
\end{definition}

\begin{example}
Зокрема маємо кілька прикладів:
\begin{enumerate}[nosep,wide=0pt,label={\arabic*)}]
\item категорія $\Ab$ буде повною підкатегорією $\Grp$;
\item категорія $\FinSet$ буде повною підкатегорією $\Set$.
\end{enumerate}
\end{example}

\begin{definition}
Категорія $C$ називається \textbf{малою}, якщо
\begin{align*}
\text{класи } \text{Ob}(C), \text{Hom}(C) \text{ -- множини.}
\end{align*}
Інакше категорія $C$ називатиметься \textbf{великою}.\\
Категорія $C$ називається \textbf{локально малою}, якщо
\begin{align*}
\text{для кожних двох об'єктів } x,y \text{ клас } C(x,y) \text{ -- множина}
\end{align*}
\end{definition}

\begin{example}
Зокрема $\Set$, $\Grp$ -- великі категорії, але локально малі.
\end{example}

\subsection{Узагальнення ін'єкції та сюр'єкції}
\subsubsection{Монік}
\begin{definition}
Задано $C$ -- категорія.\\
Морфізм $\alpha \colon x \to y$ називається \textbf{моніком}, якщо
\begin{align*}
\alpha \beta_1 = \alpha \beta_2 \implies \beta_1 = \beta_2
\end{align*}
Тобто морфізм -- монік, якщо можна завжди скоротити зліва.
\begin{figure}[H]
\centering
\begin{tikzcd}
z \arrow[r, shift left, "\beta_2"] \arrow[r, shift right, "\beta_1" {yshift=-10pt}] & x \arrow{r}{\alpha} & y
\end{tikzcd}
\end{figure}
\end{definition}

\begin{theorem}
У конкретній категорії кожний ін'єктивний морфізм -- монік.
\end{theorem}

\begin{proof}
Нехай $C$ -- конкретна категорія та $\alpha \colon X \to Y$ -- ін'єктивний морфізм. Нехай $\beta_1, \beta_2 \colon Z \to X$ -- морфізми $C$ та припустимо, що $\alpha \beta_1 = \alpha \beta_2$. Для всіх $z \in Z$ ми маємо $\alpha(\beta_1(z)) = \alpha \beta_1 (z) = \alpha \beta_2(z) = \alpha(\beta_2(z))$, тому за ін'єктивністю, $\beta_1(z) = \beta_2(z)$. Отже, $\beta_1 = \beta_2$.
\end{proof}

\begin{remark}
Зворотне твердження не працює.
\end{remark}

\begin{example}
Розглянемо повну категорію $C = \text{Div}$ підкатегорії $\text{Grp}$. Тут абелева група називається \textbf{подільною}, якщо $\forall a \in A, \forall n \in \mathbb{Z} \setminus \{0\}: \exists b \in A: a = nb$.\\
Оберемо об'єкти $\mathbb{Q}, \mathbb{Q}/_{\mathbb{Z}}$ із нашої категорії $C$ та гомоморфізм $\alpha \colon \mathbb{Q} \to \mathbb{Q}/_{\mathbb{Z}}$, який є сюр'єктивним. Даний морфізм не ін'єктивний, оскільки $\ker \alpha = \mathbb{Z}$. Стверджується, що $\alpha$ -- монік.\\
Нехай $\beta_1,\beta_2 \colon A \to \mathbb{Q}$ -- морфізми в $C$ та припустимо, що $\beta_1 \neq \beta_2$. Тоді існує елемент $a \in A$, для якого $\beta_1(a) - \beta_2(a) \neq 0$. Ліворуч раціональне число, тож $\beta_1(a) - \beta_2(a) = \dfrac{r}{s}$ для деяких $r,s \in \mathbb{Z}$ та $r \neq 0, s \neq 0$. Оскільки $A$ -- подільна група, то існує для елемента $a \in A$ та $n = 2r$ існує $b \in A$, для якого $a = nb$. Тоді $\beta_1(nb) - \beta_2(nb) = n \beta_1(b) - n \beta_2(b) = \dfrac{r}{s}$.\\
Отже, $\beta_1(b) - \beta_2(b) = \dfrac{1}{2s} \notin \mathbb{Z}$, а тому звідси $\alpha \beta_1 \neq \alpha \beta_2$.
\end{example}

\begin{theorem}
У категоріях $\text{Set},\text{Top},\text{Grp},\text{Rng}$ морфізм ін'єктивний $\iff$ морфізм -- монік.
\end{theorem}

\begin{proof}
Ми вже знаємо, що ін'єктивний морфізм -- монік. Залишилося довести зворотний бік для цих категоріях.
\bigskip \\
(\text{Set}). Нехай $\alpha \colon X \to Y$ -- монік морфізм. Оберемо $x_1,x_2 \in X$ та припустимо, що $\alpha(x_1) = \alpha(x_2)$. Покладемо $z = 0 \in \mathbb{Z}$ та покладемо $Z = \{z\}$ (хоча тут може бути будь-який сінглтон), визначимо $\beta_1, \beta_2 \colon Z \to X$ як $\beta_1(z) = x_1, \beta_2(z) = x_2$. Тоді\\
$\alpha \beta_1(z) = \alpha(\beta_1(z)) = \alpha(x_1) = \alpha(x_2) = \alpha(\beta_2(z)) = \alpha \beta_2(z)$.\\
За монічністю, звідси $\beta_1 = \beta_2$, тобто $x_1 = \beta_1(z) = \beta_2(z) = x_2$. Таким чином, $\alpha$ -- ін'єктивний.
\bigskip \\
(\text{Top}). Насправді, все аналогічно, тільки є деякі зауваження. На множину $Z$ треба задати дискретну топологію (єдина можлива топологія для неї). Відображення $\beta_1,\beta_2$ будуть уже неперервними через дискретність $Z$.
\bigskip \\
(\text{Grp}). Нехай $\alpha \colon G \to H$ -- монік морфізм. Розглянемо $\beta_1, \beta_2 \colon \ker \alpha \to G$ -- перший буде вкладенням, другий буде тривіальним. Тоді $\alpha \beta_1 = \alpha \beta_2$. Дійсно,\\
$\alpha \beta_1(g) = \alpha (g) \overset{g \in \ker \alpha}{=} e = \alpha(e) = \alpha \beta_2(g)$.\\
За монічністю, звідси $\beta_1 = \beta_2$, тобто $\beta_1$ -- тривіальне вкладення. Отже, $\ker \alpha = \{e\}$, а це означає ін'єктивніть $\alpha$.
\bigskip \\
(\text{Rng}). Таке саме доведення.
\end{proof}

\subsubsection{Розщеплений монік}
\begin{definition}
Задано $C$ -- категорія.\\
Морфізм $\alpha \colon X \to Y$ називається \textbf{розщепленим моніком}, якщо
\begin{align*}
\exists \beta \colon y \to x: \beta \alpha = 1_x
\end{align*}
Морфізм -- розщеплений монік, тобто даний морфізм має лівий оборотний.
\begin{figure}[H]
\centering
\begin{tikzcd}
x \ar[looseness=4, out=150, in=240, swap, "1_x"] \arrow[r, shift right, "\alpha" {yshift=-10pt}] & y \arrow[l, shift right, dashed, "\exists \beta" {yshift = 10pt}]
\end{tikzcd}
\end{figure}
\end{definition}

\begin{theorem}
Кожний розщеплений монік -- монік.
\end{theorem}

\begin{proof}
Нехай $\alpha \colon x \to y$ -- розщеплений монік в категорії, тобто існує морфізм $\beta \colon y \to x$, для якого $\beta \alpha = 1_x$. Нехай $\beta_1,\beta_2 \colon z \to x$ будуть морфізмами та припустимо, що $\alpha \beta_1 = \alpha \beta_2$. Тоді \\
$\beta_1 = 1_x \beta_1 = \beta \alpha \beta_1 = \beta \alpha \beta_2 = 1_x \beta_2 = \beta_2$.
\end{proof}

\begin{theorem}
У конкретній категорії кожний розщеплений монік -- ін'єктивний морфізм.
\end{theorem}

\begin{proof}
Нехай $C$ -- конкретна категорія та $\alpha \colon X \to Y$ -- розщеплений монік, тобто існує морфізм $\beta \colon Y \to X$, для якого $\beta \alpha = 1_X$. Тоді\\
$x_1 = 1_X (x_1) = \beta \alpha(x_1) = \beta(\alpha(x_1)) = \beta(\alpha(x_2)) = \beta \alpha(x_2) = 1_X(x_2) = x_2$.
\end{proof}

\begin{remark}
Зворотне твердження не працює.
\end{remark}

\begin{example}
Розглянемо категорію $\text{Grp}$. Вкладення $\alpha \colon 2 \mathbb{Z} \to \mathbb{Z}$ -- ін'єктивний гомоморфізм. Але це не буде розщепленим моніком.\\
!Припустимо, що все ж таки він розщеплений монік, тобто існує гомоморфізм $\beta \colon \mathbb{Z} \to 2 \mathbb{Z}$, для якого $\beta \alpha = 1_{2 \mathbb{Z}}$. Тоді $2 \beta(1) = \beta(2) = \beta(\alpha(2)) = \beta \alpha(2) = 2$, тобто $\beta(1) = 1$, але це суперечність! Просто тому що $\beta$ відображає на $2 \mathbb{Z}$.
\bigskip \\
Можна аналогічні міркування провести для категорії $\text{Rng}$.
\end{example}

\begin{example}
Розглянемо категорію $\text{Top}$. Оберемо тотожне відображення $\alpha \colon \mathbb{R} \to \mathbb{R}$, де область визначення має дискретну топологія, а область значень -- стандартну. Тоді $\alpha$ -- ін'єктивний, але не розщеплений монік.\\
!Припустимо, що існує морфізм $\beta \colon \mathbb{R} \to \mathbb{R}$, для якого $\beta \alpha = 1_\mathbb{R}$. Тоді $\beta = \beta 1_{\mathbb{R}} = \beta \alpha = 1_{\mathbb{R}}$, однак множина $\{0\}$ відкрита в $\mathbb{R}$ з дискретною топологією, але не відкрита в стандартній топології. Це суперечність! Тому що $\beta$ -- неперервне відображення.
\end{example}

\begin{theorem}
Задано $\alpha \colon X \to Y$ -- морфізм в категорії $\text{Set}$.\\
$\alpha$ -- розщеплений монік $\iff \begin{cases} \alpha \text{ -- ін'єктивний} \\ X = \emptyset \implies Y = \emptyset \end{cases}$.
\end{theorem}

\begin{proof}
\rightproof Дано: $\alpha$ -- розщеплений монік. Оскільки $\text{Set}$ -- конкретна категорія, то звідси $\alpha$ -- ін'єктивний.\\
Тепер нехай $X = \emptyset$. Тоді за умовою, існує $\beta \colon Y \to X$, для якого $\beta \alpha = 1_X = 1_\emptyset$. Тоді оскільки $\beta$ -- функція, то $Y = \emptyset$.
\bigskip \\
\leftproof Дано: $\alpha$ -- ін'єктивний та $X = \emptyset \implies Y = \emptyset$.\\
Нехай $X \neq \emptyset$, тобто існує елемент $x_0 \in X$. Оскільки $\alpha$ -- ін'єктивний, то $\alpha|_{\Im \alpha} \colon X \to \Im \alpha$ буде задавати бієкцію, тож для кожного $y \in \Im \alpha$ існує єдиний елемент $\beta(y) \in X$, для якого $\alpha(\beta(y)) = y$. Це визначає функцію $\beta \colon \Im \alpha \to X$, що розширюється до функції $\beta \colon Y \to X$, якщо покласти $\beta(y) = x_0, y \notin \Im \alpha$. Для $x \in X$ ми маємо $\beta \alpha(x) = \beta(\alpha(x)) = x = 1_X(x)$.\\
Нехай $X = \emptyset$, тоді $Y = \emptyset$ та порожня функція $\beta \colon Y \to X$ задовольняє $\beta \alpha = 1_X$.
\end{proof}
\noindent
Отже, в \textit{конкретній} категорії маємо таку діаграму:
\begin{align*}
\text{розщеплений монік} \implies \textit{ін'єктивний} \implies \text{монік}
\end{align*}
Приклади нам показали, що жодні два терміни не збігаються загалом.\\
У більш загальних категоріям \textit{ін'єктивність} більше не визначена, бо ми там оперуємо множинами. Але якщо слово \textit{ін'єктивний} видалити, то діаграма залишається справедливою.\\
У повній підкатегорії \text{Set}, що містить всі непорожні множини, всі ці три терміни збігаються.

\subsubsection{Епікі}
\begin{definition}
Задано $C$ -- категорія.\\
Морфізм $\alpha \colon x \to y$ називається \textbf{епіком}, якщо
\begin{align*}
\beta_1 \alpha = \beta_2 \alpha \implies \beta_1 = \beta_2
\end{align*}
Тобто морфізм -- епік, якщо можна завжди скоротити справа (дуальне означення моніка).
\begin{figure}[H]
\centering
\begin{tikzcd}
x \arrow{r}{\alpha} & y \arrow[r, shift left, "\beta_1"] \arrow[r, shift right, "\beta_2" {yshift=-10pt}] & z
\end{tikzcd}
\end{figure}
\end{definition}

\begin{theorem}
У конкретній категорії кожний сюр'єктивний морфізм -- епік.
\end{theorem}

\begin{proof}
Нехай $C$ -- конкретна категорія та $\alpha \colon X \to Y$ -- сюр'єктивний морфізм. Нехай $\beta_1,\beta_2 \colon Y \to Z$ -- морфізми $C$ та припустимо, що $\beta_1 \alpha = \beta_2 \alpha$. Оберемо $y \in Y$. Оскільки $\alpha$ -- сюр'єктивне, то $y = \alpha(x)$ для деякого $x \in X$. Тоді маємо $\beta_1(y) = \beta_1(\alpha(x)) = \beta_1 \alpha(x) = \beta_2 \alpha(x) = \beta_2(\alpha(x)) = \beta_2(y)$. Отже, $\beta_1 = \beta_2$.
\end{proof}

\begin{remark}
Зворотне твердження не працює.
\end{remark}

\begin{example}
Розглянемо категорію $\text{Rng}$ та оберемо вкладення $\alpha \colon \mathbb{Z} \to \mathbb{Q}$, яке не є сюр'єктивним. Але доведемо, що $\alpha$ -- епік.\\
Нехай $\beta_1,\beta_2 \colon \mathbb{Q} \to \mathbb{R}$ -- морфізми з $\text{Rng}$ та припустимо, що $\beta_1 \alpha = \beta_2 \alpha$. Тоді $\beta_1(n) = \beta_2(n)$ для будь-якого цілого $n \in \mathbb{Z}$. При $n \neq 0$ ми маємо \\
$\beta_1(n^{-1}) = \beta_1(n^{-1} \cdot 1) = \beta_1(n^{-1})\beta_1(1) = \beta_1(n^{-1}) \beta_2(1) = \beta_1(n^{-1})\beta_2(n)\beta_2(n^{-1}) = \beta_1(n^{-1})\beta_1(n)\beta_2(n^{-1}) = \beta_1(1)\beta_2(n^{-1}) = \beta_2(1)\beta_2(n^{-1}) = \beta_2(1 \cdot n^{-1}) = \beta_2(n^{-1})$.\\
Таким чином, для $m,n \in \mathbb{Z}$ при $n \neq 0$ ми маємо наступне:\\
$\beta_1\left( \dfrac{m}{n} \right) = \beta_1(m) \beta_1(n^{-1}) = \beta_2(m) \beta_2(n^{-1}) = \beta_2(m) \beta_2(n^{-1}) = \beta_2\left( \dfrac{m}{n} \right)$.\\
Отже, $\beta_1 = \beta_2$.
\end{example}

\begin{theorem}
У категорія $\text{Set}, \text{Top}, \text{Grp}$ морфізм сюр'єктивний $\iff$ морфізм -- епік.
\end{theorem}

\begin{proof}
Ми вже знаємо, що сюр'єктивний морфізм -- епік. Залишилося довести зворотний бік для цих категоріях.
\bigskip \\
(Set). Нехай $\alpha \colon X \to Y$ -- епік морфізм. Нехай $\beta_1 \colon Y \to \{0,1\}$ буде характеристичною функцією для $\Im \alpha$ та нехай $\beta_2 \colon Y \to \{0,1\}$ буде стало дорівнювати $1$. Тоді $\beta_1 \alpha = \beta_2 \alpha$, тому за епічністю, $\beta_1 = \beta_2$. Із цього випливає, що $\Im \alpha = Y$, що доводить сюр'єктивність $\alpha$.
\bigskip \\
(Top). Проводиться те саме доведення, як з Set. Тільки треба $\alpha \colon X \to Y$ брати уже неперервне відображення, а на просторі $\{0,1\}$ задати недискретну топологію, щоб $\beta_1,\beta_2$ стали неерервними.
\bigskip \\
(Grp). !Нехай $\alpha \colon G \to H$ -- гомоморфізм груп та припустимо, що це -- не сюр'єктивний. Звідси випливає, що $[H:\Im \alpha] > 1$. Ми тоді доведемо, що $\alpha$ -- не епік морфізм.\\
Випадок $[H:\Im \alpha] = 2$. Нехай $\beta_1 \colon H \to H/_{\Im \alpha}$ -- канонічний гомоморфізм та $\beta_2 \colon H \to H/_{\Im \alpha}$ -- тривіальний гомоморфізм. Тоді $\beta_1 \alpha = \beta_2 \alpha$, але при цьому $\beta_1 \ne q\beta_2$, оскільки $\Im \alpha \neq H$. Тобто в даному випадку $\alpha$ -- не епік.\\
Випадок $[H:\Im \alpha] > 2$. Тоді існують два різних правих суміжних класи $K_1 = \Im \alpha \cdot h_1$ та $K_2 = \Im \alpha \cdot h_2$, причому $K_1,K_2 \neq \Im \alpha$. Покладемо $b = h_1^{-1}h_2$ та зауважимо, що $K_1b = K_2$, а звідси $K_2b^{-1} = K_1$. Позначимо $S_H$ за групу симетрії на $H$ та оберемо бієкцію $\sigma \in S_H$, що задана формулою $\sigma(h) = \begin{cases} hb, & h \in K_1, \\ hb^{-1}, & h \in K_2, \\ h, & \text{інакше} \end{cases}$. Можна зауважити, що $\sigma^2 = 1_H$ та $\sigma(kh) = k \sigma(h)$ для всіх $k \in \Im \alpha, h \in H$. Для $h \in H$ нехай $\lambda_h$ буде елементом $S_H$, що заданий формулою $\lambda_h(x) = hx (x \in H)$. Тоді звідси отримаємо $\sigma \lambda_k = \lambda_k \sigma$ для всіх $k \in \Im \alpha$.\\
Визначимо $\beta_1,\beta_2 \colon H \to S_H$ як $\beta_1(h) = \lambda_k$ та $\beta_2(h) = \sigma \lambda_k \sigma
$. Ці два відображення справдлі задають гомоморфізм груп. Для $k \in \Im \alpha$ ми маємо\\
$\beta_2(k) = \sigma \lambda_k \sigma = \lambda_k \sigma^2 = \lambda_k = \beta_1(k)$, а тому $\beta_1 \alpha = \beta_2 \alpha$. Із іншого боку, $\beta_2(h_1)(e) = \sigma \lambda_{h_1} \sigma (e) = \sigma(h_1) = h_2 \neq h_1 = \lambda_{h_1}(e) = \beta_1(h_1)(e)$. Тож звідси $\beta_1 \neq \beta_2$. Тобто і в цьому випадку $\alpha$ -- не епік.
\end{proof}

\subsubsection{Розщеплений епік}
\begin{definition}
Задано $C$ -- категорія.\\
Морфізм $\alpha \colon X \to Y$ називається \textbf{розщепленим епіком}, якщо
\begin{align*}
\exists \beta \colon y \to x: \alpha \beta = 1_y
\end{align*}
Морфізм -- розщеплений епік, тобто даний морфізм має правий оборотний (дуальне означення розщепленого моніка). Такий морфізм інколи ще називають \textbf{ретракцією}.
\begin{figure}[H]
\centering
\begin{tikzcd}
x \arrow[r, shift right, "\alpha" {yshift=-10pt}] & y \arrow[l, shift right, dashed, "\exists \beta" {yshift = 10pt}] \ar[looseness=4, out=-30, in=60, swap, "1_y"]
\end{tikzcd}
\end{figure}
\end{definition}

\begin{theorem}
Кожний розщеплений епік -- епік.
\end{theorem}

\begin{proof}
Нехай $\alpha \colon x \to y$ -- розщеплений епік в категорії, тобто існує морфізм $\beta \colon y \to x$, для якого $\alpha \beta = 1_1$. Нехай $\beta_1,\beta_2 \colon y \to z$ будуть морфізмами та припустимо, що $\beta_1 \alpha = \beta_2 \alpha$. Тоді \\
$\beta_1 = \beta_1 1_y = \beta_1 \alpha \beta = \beta_2 \alpha \beta = \beta_2 1_y = \beta_2$.
\end{proof}

\begin{theorem}
У конкретній категорії кожний розщеплений епік -- сюр'єктивний морфізм.
\end{theorem}

\begin{proof}
Нехай $C$ -- конкретна категорія та $\alpha \colon X \to Y$ -- розщеплений епік, тобто існує морфізм $\beta \colon Y \to X$, для якого $\alpha \beta = 1_Y$. Нехай $y \in Y$, тоді покладемо $x = \beta(y)$. Звідси\\
$\alpha(x) = \alpha(\beta(y)) = \alpha \beta(y) = 1_Y(y) = y$.
\end{proof}

\begin{remark}
Зворотне твердження не працює.
\end{remark}

\begin{example}
Розглянемо категорію $\text{Grp}$ та визначимо морфізм $\alpha \colon \mathbb{Z}_4 \to \mathbb{Z}_2$, визначений як $\alpha(0) = \alpha(2) = 0$ та $\alpha(1) = \alpha(3) = 1$. Це буде сюр'єктивний гомоморфізм. Оскільки $1 \in \mathbb{Z}_2$ має порядок $2$, то будь-який гомоморфізм $\beta \colon \mathbb{Z}_2 \to \mathbb{Z}_4$ зобов'язаний відображати $1$ на $0$ або $2$. Таким чином, $\alpha \beta \neq 1_{\mathbb{Z}_2}$. Отже, $\alpha$ -- не розщеплений епік.
\bigskip \\
Можна аналогічні міркування провести для категорії Rng.
\end{example}

\begin{example}
Розглянемо категорію Top. Маємо $\alpha \colon \mathbb{R} \to \mathbb{R}$ -- тотожне відображення; у першого -- дискретна топологія, у другого -- стандартна. Тоді $\alpha$ -- сюр'єктивний морфізм, але аналогічним чином можна довести, що це не епічний морфізм (як це було з епічним моніком).
\end{example}

\begin{theorem}
У категорії $\text{Set}$ морфізм -- розщеплений епік $\iff$ морфізм сюр'єктивний.
\end{theorem}

\begin{proof}
Залишилося довести у зворотний бік.\\
\leftproof Дано: $\alpha \colon X \to Y$ -- сюр'єктивний морфізм. Тобто для кожного $y \in Y$ знайдеться $\beta(y) \in X$, для якого $\alpha(\beta(y)) = y$, а це визначає функцію $\beta \colon Y \to X$, яка задовольняє $\alpha \beta = 1_Y$. Отже, $\alpha$ -- розщеплений епік.
\end{proof}
\noindent
Отже, в \textit{конкретній} категорії маємо таку діаграму:
\begin{align*}
\text{розщеплений епік} \implies \textit{сюр'єктивний} \implies \text{епік}
\end{align*}
Приклади нам показали, що жодні два терміни не збігаються загалом.\\
У більш загальних категоріям \textit{сюр'єктивність} більше не визначена, бо ми там оперуємо множинами. Але якщо слово \textit{сюр'єктивний} видалити, то діаграма залишається справедливою.\\
У категорії \text{Set} всі ці три терміни збігаються.

\subsubsection{Біморфізми та ізоморфізми}
\begin{definition}
Задано $C$ -- категорія.\\
Морфізм $\alpha \colon x \to y$ називається \textbf{біморфізмом}, якщо
\begin{align*}
\alpha \text{ -- одночасно монік та епік}
\end{align*}
Морфізм $\alpha \colon x \to y$ називається \textbf{ізоморфізмом}, якщо
\begin{align*}
\exists \beta \colon y \to x: \beta \alpha = 1_x \qquad \alpha \beta = 1_y
\end{align*}
\end{definition}

\begin{remark}
Якщо $\alpha$ -- ізоморфізм, то морфізм $\beta$ в означенні -- єдиний та позначається за $\alpha^{-1}$.
\end{remark}

\begin{definition}
Задано $C$ -- категорія.\\
Об'єкти $x,y$ називаються \textbf{ізоморфними}, якщо
\begin{align*}
\exists \alpha \colon x \to y \text{ -- ізоморфізм}
\end{align*}
Позначення: $x \cong y$ (це справді відношення еквівалентності).
\end{definition}

\begin{theorem}
Морфізм -- ізоморфізм $\iff$ морфізм -- розщеплений монік та розщеплений епік.
\end{theorem}

\begin{proof}
\rightproof \textit{миттєво випливає з означення.}
\bigskip \\
\leftproof Дано: $\alpha$ -- розщеплений монік та розщеплений епік. Тобто існують морфізми $\beta, \gamma \colon y \to x$, для яких $\beta \alpha = 1_x, \quad \alpha \gamma = 1_y$. Але тоді $\beta = \beta 1_y = \beta \alpha \gamma = 1_x \gamma = \gamma$. Отже, $\alpha$ -- ізоморфізм.
\end{proof}
\noindent Тепер ми маємо ось таку діаграму. Італік позначений лише для конкретних категорій.
\begin{figure}[H]
\centering
\begin{tikzcd}
\text{монік} & & \text{епік} \\
& \arrow[Rightarrow] {lu} \text{біморфізм} \arrow[Rightarrow] {ru} & \\
\arrow[Rightarrow] {uu} \textit{ін'єктивний} & & \textit{сюр'єктивний} \arrow[Rightarrow] {uu} \\
& \arrow[Rightarrow] {lu} \textit{бієкція} \arrow[Rightarrow] {ru} \arrow[Rightarrow] {uu} & \\
\arrow[Rightarrow] {uu}
\text{розщеплений монік} & & \text{розщеплений епік} \arrow[Rightarrow] {uu} \\
& \arrow[Rightarrow] {lu} \text{ізоморфізм} \arrow[Rightarrow] {ru} \arrow[Rightarrow] {uu} & \\
\end{tikzcd}
\end{figure}

\begin{theorem}
У категорії Set, Grp біморфізм, бієкція, ізоморфізм -- це одне й те саме.
\end{theorem}

\begin{proof}
(Set). Нехай $\alpha \colon X \to Y$ -- біморфізм. Зважаючи на діаграму вище, достатньо довести, що $\alpha$ -- ізоморфізм. Оскільки $\alpha$ -- монік та епік, то в даній категорії $\alpha$ -- ін'єктивний та сюр'єктивний, тобто бієктивний. Значить, існує морфізм $\alpha^{-1}$, для якого $\alpha^{-1} \alpha = 1_X, \quad \alpha \alpha^{-1} = 1_Y$, що й доводить ізоморфність.
\bigskip \\
(Grp). Насправді, аналогічно. Але треба окремо пересвідчитися, що якщо $\alpha$ -- гомоморфізм, то $\alpha^{-1}$ буде ним також.
\end{proof}

\begin{remark}
Що по інших категоріях, які не потрапили в цю теорему.\\
(Rng). Зауважимо, що $\mathbb{Z} \hookrightarrow \mathbb{Q}$ буде біморфізмом, але не бієкцією.\\
(Top). Тотожне відображення $R \to R$, з дискретною та стандартною топологією відповідно, буде бієкцією, але не ізоморфізмом (тобто гомеоморфізмом в даному випадку).
\end{remark}
\end{document}